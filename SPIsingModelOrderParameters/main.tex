\documentclass[aps, 12pt]{revtex4-2}

\usepackage{amsmath}
\usepackage{physics}

\begin{document}
\title{Note: Order parameters of the Ising model for ferromegnetism}
\author{Hironao Yamato}
\affiliation{Department of Physics, Osaka University}
\date{\today}
\maketitle

\section{Magnetization and magnetic susceptibility}
In dealing with the Ising model for ferromagnetism, let us first define magnetization and magnetic susceptibility. Magnetization, which is often denoted as $M$, is defined as follows.
\begin{align}
  M \equiv \sum_{i} \ev{S_i}.
\end{align}
Since $M$ is a extensive quantity, as you can see from its definition, it is preferable to define $m$, which is the magntization per site.
\begin{align}
  m \equiv \frac{M}{N} = \frac{1}{N} \sum_i \ev{S_i}.
\end{align}
As thermodynamics suggests, $m$ can be calculated using the following relation.
\begin{align}
  m = -\pdv{g}{h},
\end{align}
where $g$ is the free energy per site of the system under consideration. This can be easily proven without deviating the framework of the canonical ensemble:
\begin{align}
  g            &= -\frac{1}{\beta N} \log Z = -\frac{1}{\beta N} \log \qty{ \sum \exp(\beta J \sum_{(i, j)}S_iS_j + \beta h \sum_i S_i)} \quad \text{and} \\
  - \pdv{g}{h} &= \frac{1}{N} \sum \qty{\qty(\sum_i S_i)\frac{1}{Z}\exp(\beta J \sum_{(i, j)}S_iS_j + \beta h \sum_i S_i)} = \frac{1}{N} \sum_i \ev{S_i} = m.
\end{align}

Magnetic susceptibility, which is often denoted as $\chi$, is defined as follows.
\begin{align}
  \chi \equiv \frac{\beta}{N}\qty{\ev{\qty(\sum_iS_i)^2} - \ev{\sum_iS_i}^2}.
\end{align}
It is worthwhile to know that $\chi$ can be written in the following form.
\begin{align}
  \chi &= \frac{\beta}{N}\qty{\ev{\qty(\sum_i S_i)\qty(\sum_j S_j)} - \qty(\sum_i\ev{S_i})\qty(\sum_j\ev{S_j})} \\
       &= \frac{\beta}{N}\qty{\sum_i\sum_j \ev{S_iS_j} - \qty(\sum_i\ev{S_i})\qty(\sum_j\ev{S_j})}              \\
       &= \frac{\beta}{N}\sum_i\sum_j \qty{\ev{S_iS_j} - \ev{S_i}\ev{S_j}}
\end{align}
\bibliography{basename of .bib file}
\end{document}
