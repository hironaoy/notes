\documentclass[aps, 12pt]{revtex4-2}

%% \usepackage{mathrsfs}
\usepackage{physics}
\usepackage{breqn}
\usepackage{amsmath, amssymb, amsthm}

\begin{document}
\title{Note: Proof of Lee-Yang Theorem}
\author{Hironao Yamato}
\date{\today}
\maketitle
\section{Lee-Yang Theorem}

\subsection{The setup and lemmas}

\subsection{Generalised proposition and its proof}
Let
\begin{dmath}
 \varXi^{{G}}(\qty{z_j}) = \sum_{\qty{n_j}}\prod_{j}\qty(z_{j})^{n_j}e^{-4\beta\sum_{(j<k)}V_{jk}(n_j - n_k)^2} \quad \text{(grand partition function).} \label{gpf}
\end{dmath}
Then the grand partition function $\varXi^{{G}}{(\qty{z_j})}$ is non-zero in ${\displaystyle \bigcap_{j = 1}^{N}}\qty{\qty|z_j| < 1}$. \\
This proposition can be proven by induction on $N$, which is the number of sites of the system under consideration. \textbf{Supose this proposition is true for the situation which the number of sites are less than or equal to $N - 1$.} (hypothesis) The grand partition function can be written in the following form by calculating the sum over $n_N$ in \eqref{gpf}:
\begin{dmath}
  \varXi^{G}(\qty{z_j}) = \sum_{n_N = 0, 1}(z_N)^{n_N} \sum_{\qty{n_j;\, j < N}} \prod_{j < N}\qty(z_j)^{n_j} e^{-4\beta \sum_{k < N}V_{1k}(n_j - n_k)^2} e^{-4\beta\sum_{(j<k<N)}(n_j - n_k)^2} \\
                        = \sum_{\qty{n_j;\, j < N}} \prod_{j < N}\qty(z_j)^{n_j} e^{-4\beta \sum_{k<N}V_{1k}n_k^2} e^{-4\beta\sum_{(j<k<N)}(n_j - n_k)^2} + z_N\sum_{\qty{n_j;\, j < N}} \prod_{j < N}\qty(z_j)^{n_j} e^{-4\beta \sum_{k < N}V_{1k}(1 - n_k)^2} e^{-4\beta\sum_{(j<k<N)}(n_j - n_k)^2}.
\end{dmath}
Since $n_k$ only takes $0$ or $1$,
\begin{align}
  e^{-4\beta \sum_{k<N}V_{1k}n_k^2} = e^{-4\beta \sum_{k<N}V_{1k}n_k^2}, \quad e^{-4\beta \sum_{k<N}V_{1k}(1 - n_k)^2} = e^{4\beta \sum_{k<N}V_{1k}n_k - 4\beta\sum_{k<N}V_{1k}}.
\end{align}
Then, setting $a_{jk} := e^{-\beta V_{jk}}$,
\begin{dmath}
  \varXi^{G}(\qty{z_j}) = \varXi^{G\backslash \qty{N}}(a_{12}z_2, a_{13}z_3, \dots) + z_1a_{12}a_{13}\cdots a_{1N}\varXi^{G\backslash \qty{N}}(z_{2}/a_{12}, z_2/a_{13}, \cdots).
\end{dmath}
where $\varXi^{G\backslash \qty{N}}(\qty{z_j})$ is the grand partition function for a system with $N - 1$ sites. In other words, $\varXi^{G\backslash \qty{N}}(\qty{z_j})$ is the grand partition function for the system that the $N$th site is removed. Then using \eqref{id-xi} and \eqref{id-star}, we can write
\begin{dmath}
  \varXi^{G}(\qty{z_j}) = \varXi^{G\backslash \qty{N}}(a_{12}z_2, a_{13}z_3, \dots) + z_1z_2 \cdots z_N\varXi^{G\backslash \qty{N}}(a_{12}/z_{2}^*, a_{13}/z_2^*, \cdots). \label{gpf-expanded}
\end{dmath}
What we need to show is that $\varXi^{G}(\qty{z_j})$ is non-zero. Suppose \eqref{gpf-expanded} be zero, then at least the following condition must be met.
\begin{dmath}
  \qty|\varXi^{G\backslash \qty{N}}(a_{12}z_2, a_{13}z_3, \dots)| = \qty|z_1z_2 \cdots z_N\varXi^{G\backslash \qty{N}}(a_{12}/z_{2}^*, a_{13}/z_2^*, \cdots)|
\end{dmath}
Since $\varXi^{G\backslash \qty{N}}(a_{12}z_2, a_{13}z_3, \dots)$ is non-zero by hypothesis, this condition can be rewritten in the following form.
\begin{dmath}
  \qty|\frac{z_1z_2 \cdots z_N\varXi^{G\backslash \qty{N}}(a_{12}/z_{2}^*, a_{13}/z_2^*, \cdots)}{\varXi^{G\backslash \qty{N}}(a_{12}z_2, a_{13}z_3, \dots)}| = 1
\end{dmath}

%% \bibliography{mainNotes}
\end{document}
