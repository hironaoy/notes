\documentclass[aps, 12pt]{revtex4-2}

%% \usepackage{mathrsfs}
\usepackage{physics}
\usepackage{breqn}
\usepackage{amsmath, amssymb, amsthm}

\begin{document}
\title{Note: Proof of Lee-Yang Theorem \footnote{See: \texttt{https://github.com/hironaoy/notes/LeeYangTheorem}}}
\author{Hironao Yamato}

\date{\today}
\maketitle
\section{The setup and lemmas}
We will be dealing with the following grand partition function in this document.
\begin{align}
  \varXi^{{G}}(\qty{z_j}) = \sum_{\qty{n_j = 0, 1}}\prod_{j}\qty(z_{j})^{n_j}e^{-4\beta\sum_{(j<k)}V_{jk}(n_j - n_k)^2} \quad \text{(grand partition function)}\label{gpf}
\end{align}
This grand partition function is for Ising model or lattice gas model.

In order to facilitate the discussion in the next section, let us first introduce two identities and a theorem.

\noindent
\textbf{Lemma.} \, The grand partition function \eqref{gpf} satisfies the following identity.
\begin{align}
  \varXi^G(\qty{z_j}) = z_1z_2 \cdots z_N\varXi^G(\qty{1/z_j})
\end{align}
\underline{Proof.}\quad By simply calculating the right-hand side, 
\begin{align}
  z_1z_2 \cdots z_N \varXi^G(\qty{1/z_j}) &= z_1z_2 \cdots z_N \sum_{\qty{n_j = 0, 1}} \prod_{j = 1}^N\qty(\frac{1}{z_j})^{n_j} e^{-4\beta\sum_{(j<k)}V_{jk}(n_j - n_k)^2} \nonumber \\
  &= \sum_{\qty{n_j = 0, 1}} \prod_{j = 1}^N z_j^{1 - n_j} e^{-4\beta\sum_{(j<k)}V_{jk}(n_j - n_k)^2} \nonumber \\
  &= \sum_{\qty{m_j = 1, 0}} \prod_{j = 1}^N z_j^{m_j} e^{-4\beta\sum_{(j<k)}V_{jk}(m_j - m_k)^2} \quad \qty(m_j := 1 - m_j)
\end{align}
Here, $nj$ and $m_j$ are dummy index, then we are done. \hfill \rule{1.5mm}{3.5mm}

\noindent
\textbf{Lemma.}\quad The grand partition function \eqref{gpf} satisfies the following identity.
\begin{align}
  \varXi^G(\qty{z_j}) = \qty{\varXi^G(\qty{z_j^*})}^*
\end{align}
\underline{Proof.}\quad Obivous.
\section{Generalised proposition and its proof}
\noindent
\textbf{Proposition.} \, Let
\begin{dmath}
 \varXi^{{G}}(\qty{z_j}) = \sum_{\qty{n_j = 0, 1}}\prod_{j}\qty(z_{j})^{n_j}e^{-4\beta\sum_{(j<k)}V_{jk}(n_j - n_k)^2} \quad \text{(grand partition function).} \label{gpf}
\end{dmath}
Then the grand partition function $\varXi^{{G}}{(\qty{z_j})}$ is non-zero in ${\displaystyle \bigcap_{j = 1}^{N}}\qty{\qty|z_j| < 1}$. \\
\underline{Proof.}\quad This proposition can be proven by induction on $N$, which is the number of sites of the system under consideration. \textbf{Suppose this proposition is true for the situation which the number of sites are less than or equal to $N - 1$.} The grand partition function can be written in the following form by calculating the sum over $n_N$ in \eqref{gpf}:
\begin{dmath}
  \varXi^{G}(\qty{z_j}) = \sum_{n_N = 0, 1}(z_N)^{n_N} \sum_{\qty{n_j;\, j < N}} \prod_{j < N}\qty(z_j)^{n_j} e^{-4\beta \sum_{k < N}V_{1k}(n_j - n_k)^2} e^{-4\beta\sum_{(j<k<N)}(n_j - n_k)^2} \\
                        = \sum_{\qty{n_j;\, j < N}} \prod_{j < N}\qty(z_j)^{n_j} e^{-4\beta \sum_{k<N}V_{1k}n_k^2} e^{-4\beta\sum_{(j<k<N)}(n_j - n_k)^2} + z_N\sum_{\qty{n_j;\, j < N}} \prod_{j < N}\qty(z_j)^{n_j} e^{-4\beta \sum_{k < N}V_{1k}(1 - n_k)^2} e^{-4\beta\sum_{(j<k<N)}(n_j - n_k)^2}.
\end{dmath}
Since $n_k$ only takes $0$ or $1$,
\begin{align}
  e^{-4\beta \sum_{k<N}V_{1k}n_k^2} = e^{-4\beta \sum_{k<N}V_{1k}n_k^2}, \quad e^{-4\beta \sum_{k<N}V_{1k}(1 - n_k)^2} = e^{4\beta \sum_{k<N}V_{1k}n_k - 4\beta\sum_{k<N}V_{1k}}.
\end{align}
Then, setting $a_{jk} := e^{-\beta V_{jk}} \leq 1$,
\begin{dmath}
  \varXi^{G}(\qty{z_j}) = \varXi^{G\backslash \qty{N}}(a_{12}z_2, a_{13}z_3, \dots) + z_1a_{12}a_{13}\cdots a_{1N}\varXi^{G\backslash \qty{N}}(z_{2}/a_{12}, z_2/a_{13}, \cdots).
\end{dmath}
where $\varXi^{G\backslash \qty{N}}(\qty{z_j})$ is the grand partition function for a system with $N - 1$ sites. In other words, $\varXi^{G\backslash \qty{N}}(\qty{z_j})$ is the grand partition function for the system that the $N$th site is removed. Then using \eqref{id-xi} and \eqref{id-star}, we can write
\begin{dmath}
  \varXi^{G}(\qty{z_j}) = \varXi^{G\backslash \qty{N}}(a_{12}z_2, a_{13}z_3, \dots) + z_1z_2 \cdots z_N\varXi^{G\backslash \qty{N}}(a_{12}/z_{2}^*, a_{13}/z_2^*, \cdots). \label{gpf-expanded}
\end{dmath}
What we need to show is that $\varXi^{G}(\qty{z_j})$ is non-zero. Suppose \eqref{gpf-expanded} be zero, then at least the following condition must be met.
\begin{dmath}
  \qty|\varXi^{G\backslash \qty{N}}(a_{12}z_2, a_{13}z_3, \dots)| = \qty|z_1z_2 \cdots z_N\varXi^{G\backslash \qty{N}}(a_{12}/z_{2}^*, a_{13}/z_2^*, \cdots)|
\end{dmath}
Since $\varXi^{G\backslash \qty{N}}(a_{12}z_2, a_{13}z_3, \dots)$ is non-zero by hypothesis, this condition can be rewritten in the following form.
\begin{dmath}
  \qty|\frac{z_1z_2 \cdots z_N\varXi^{G\backslash \qty{N}}(a_{12}/z_{2}^*, a_{13}/z_2^*, \cdots)}{\varXi^{G\backslash \qty{N}}(a_{12}z_2, a_{13}z_3, \dots)}| = 1 \label{unitcirc}
\end{dmath}
Now, let us prove that this condition is not met by proving the sufficient condition:
\begin{dmath}
  \sup_{\qty|a_{1j}| \leq 1} \sup_{\qty|z_j| < 1} \qty|\frac{z_1z_2 \cdots z_N\varXi^{G\backslash \qty{N}}(a_{12}/z_{2}^*, a_{13}/z_2^*, \cdots)}{\varXi^{G\backslash \qty{N}}(a_{12}z_2, a_{13}z_3, \dots)}| < 1. \label{suffic-con}
\end{dmath}
By the maximum modulas principle, which was introduced in the last section, we only need to consider the boundary with respect to $z_j$.
\begin{align}
  \sup_{\qty|a_{1j}| \leq 1} \sup_{\qty|z_j| < 1} \qty|\frac{z_1z_2 \cdots z_N\varXi^{G\backslash \qty{N}}(a_{12}/z_{2}^*, a_{13}/z_2^*, \cdots)}{\varXi^{G\backslash \qty{N}}(a_{12}z_2, a_{13}z_3, \dots)}| &< \sup_{\qty|a_{1j}| \leq 1} \sup_{\qty|z_j| = 1} \qty|\frac{z_1z_2 \cdots z_N\varXi^{G\backslash \qty{N}}(a_{12}/z_{2}^*, a_{13}/z_2^*, \cdots)}{\varXi^{G\backslash \qty{N}}(a_{12}z_2, a_{13}z_3, \dots)}| \nonumber \\
  &= \sup_{\qty|a_{1j}| \leq 1} \sup_{\qty|z_j| < 1} \qty|\frac{\varXi^{G\backslash \qty{N}}(a_{12}/z_{2}^*, a_{13}/z_2^*, \cdots)}{\varXi^{G\backslash \qty{N}}(a_{12}z_2, a_{13}z_3, \dots)}|
\end{align}
Also, since $\qty|z_j| = 1$, we can replace $1/z_j^*$ with $z_j$. Then we get
\begin{align}
  \sup_{\qty|a_{1j}| \leq 1} \sup_{\qty|z_j| < 1} \qty|\frac{z_1z_2 \cdots z_N\varXi^{G\backslash \qty{N}}(a_{12}/z_{2}^*, a_{13}/z_2^*, \cdots)}{\varXi^{G\backslash \qty{N}}(a_{12}z_2, a_{13}z_3, \dots)}| < \sup_{\qty|a_{1j}| \leq 1} \sup_{\qty|z_j| = 1} \qty|\frac{\varXi^{G\backslash \qty{N}}(a_{12}/z_{2}^*, a_{13}/z_2^*, \cdots)}{\varXi^{G\backslash \qty{N}}(a_{12}z_2, a_{13}z_3, \dots)}| = 1.
\end{align}
Here, we proved \eqref{suffic-con} which proves the negation of \eqref{unitcirc} which proves that the grand partition function for a system with $N$ sites is non-zero under the hypothesis that grand partition function for a system with $N-1$ sites is non-zero. Then, we are done. \hfill \rule{1.5mm}{3.5mm}
\section{Lee-Yang Theorem}
Now, we specialise to the case when all the $z_j$ are $z$, and $\varXi^{G}(z)$ is a polynomial of degree $N$.

\noindent
\textbf{Theorem.} \, The zeros of $\varXi^G(z)$ all lie on the unit circle $\qty{z; \,\qty|z| = 1}$ if exist. \\
\underline{Proof.}\quad From the proposition we proved in the last section, the interior of the unit circle is free of zeros. From \eqref{id-xi}, which with the proposition leads to
\begin{align} 
  \varXi^{G}(z) = z^N \varXi^{G}(1/z) \ne 0 \quad \text{in} \quad \qty{z;\, \qty|z| < 1},
\end{align}
it is also non-zero in the exterior. In order to check that we can subsititute $1/z$ with some other variable $w$:
\begin{dmath}
  \frac{\varXi^G(w)}{w^N} \ne 0 \quad \text{in} \quad \qty{w; \, \qty|w| > 1}
\end{dmath}
We are done. \hfill \rule{1.5mm}{3.5mm}

\bibliography{mainNotes}
\end{document}
