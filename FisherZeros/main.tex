\documentclass[aps, 12pt]{revtex4-2}

\usepackage{amsmath, physics}
\begin{document}
\title{Note: Fisher zeros for two dimensional Ising model lie on unit circle in the complex $w := \sinh \beta J$ plane.}
\author{Hironao Yamato}
\date{\today}
\maketitle

\section{The setup and the proof}
The exanct expression of the partition function of a finite two dimensional Ising model is as follows.
\begin{align}
  Z = 2^N\prod_{j = 1}^{L} \prod_{k = 1}^{L} \qty{\qty(\frac{1 + z^2}{1 - z^2})^2 - \frac{2z}{1 - z^2}\qty(\cos \frac{2\pi j}{L} + \cos \frac{2\pi k}{L})}^{1/2}, \quad z := \tanh\beta J
\end{align}
Now let us rewrite this expression in terms of a new variable $w := \sinh \beta J$. Since
\begin{gather}
  \qty(\frac{1 + z^2}{1 - z^2})^2 = \qty(\frac{2}{1 - z^2} - 1)^2 = \qty(2\cosh^2 \beta J - 1)^2 = \qty(\cosh 2\beta J)^2 = 1 + \sinh^2 2\beta J, \\
  \frac{2z}{1 - z^2} = 2 \cosh ^2 \beta J \tanh \beta J = 2\cosh \beta J \sinh \beta J = \sinh 2\beta J,
\end{gather}
then
\begin{align}
  Z = 2^N\prod_{j = 1}^{L} \prod_{k = 1}^{L} \qty{1 + w^2 - w\qty(\cos \frac{2\pi j}{L} + \cos \frac{2\pi k}{L})}^{1/2}, \quad w := \sinh 2 \beta J.
\end{align}
For simplicity, we now write
\begin{align}
  a_L(j,k) := \cos \frac{2\pi j}{L} + \cos \frac{2\pi k}{L},
\end{align} 
and 
\begin{align}
  Z = 2^N\prod_{j = 1}^{L} \prod_{k = 1}^{L} \qty{w^2 - a_L(j,k)w + 1}^{1/2}, \quad w := \sinh 2 \beta J.
\end{align}
Now we can easily find the zeros of $Z$ in terms of $w$ by solving the equation
\begin{align}
  w^2 - a_L(j,k)w + 1 = 0
\end{align}
for all $j$ and $k$. Using the quadratic formula we get
\begin{align}
  w = \frac{a_L(j,k)}{2} \pm i \frac{\sqrt{4 - a^2_L(j,k)}}{2}
\end{align}
As one can easily see from the form of the solution, $w$ lie on unit circle $\qty|w| = 1$.
\begin{align}
  \qty|w|^2 = \qty{\frac{a_L(j,k)}{2}}^2 + \qty{\frac{\sqrt{4 - a_L^2(j,k)}}{2}}^2 = \frac{a_L^2(j,k)}{4} + \frac{4 - a^2_L(j,k)}{4} = 1
\end{align}

In the case of a system described by two dimensional Ising model, we can see that some singuralityies appear on the real axis, and some of the $w$ give a real inverse temperature. It can be obtained by the following procedure. First of all, the real $w$ that satisify the realation $\qty|w| = 1$ are $\pm 1$. Then
\begin{align}
  w = \sinh 2\beta J = \frac{e^{2\beta J} - e^{-2\beta J}}{2} = \pm 1 \quad \longrightarrow \quad \qty(e^{2\beta J})^2 \pm 2 e^{2\beta J} - 1= 0. \label{eqn}
\end{align}
By solving the equation above \eqref{eqn}, we get
\begin{align}
  e^{2\beta J} = \sqrt{2} \pm 1 \quad \longrightarrow \quad e^{\beta J} = \qty(\sqrt{2} \pm 1)^{1/2}
\end{align}
Here we only consider the solutions which gives the real inverse temperature. The $z = \tanh \beta J$ that corresponds to the real $w$ at singularity is as follows.
\begin{align}
  z = \tanh \beta J = \frac{e^{\beta J} - e^{-\beta J}}{e^{\beta J} + e^{-\beta J}} = \sqrt{2} - 1 \, \mathrm{or} \, 1 - \sqrt{2}.
\end{align}
According to Fisher, the former solution locate the ferromagnetic temperature and the latter locate the antiferromagnetic temperature.
%% TODO: Prove this.

\section{Comparison with the classical prediction}
%% TODO: Add the result by mean field theory
Classicaly, we can predict the transition point by taking the thermodynamic limit of the free energy per site and looking for a point where the free energy gets singular. In this case, the free energy per site in the thermodynamic limit is as follows.
\begin{align}
  f = -\frac{1}{\beta} \log \frac{2}{1 - z^2} - \frac{1}{2\beta} \int_0^{2\pi} \frac{\dd \omega_j}{2\pi} \int_0^{2\pi} \frac{\dd \omega_k}{2\pi} \log \qty{(1 + z^2)^2 - 2z(1 - z^2)\qty(\cos \omega_j + \cos \omega_k)}
\end{align}
Here the integrand is minmized, in term of $\omega_j$ and $\omega_k$, at the point where satisify $\cos \omega_j + \cos \omega_k = 2$ and
\begin{align}
  (1 + z^2)^2 - 4z(1 - z^2) = (z^2 + 2z - 1)^2 \geq 0,
\end{align}
then we get
\begin{align}
  (1 + z^2)^2 - 2z(1 - z^2)\qty(\cos \omega_j + \cos \omega_k) \geq 0.
\end{align}
This implies that at the singular points, $z^2 + 2z - 1 = 0$ has to be satisified. For ferromagnetic model, by taking the positive solution, we can predict the transition point
\begin{align}
  z = \sqrt{2} - 1
\end{align}
which is the same value as the one predicted above using the method of Lee and Yang.
\bibliography{basename of .bib file}
\end{document}
